\section{综合}

\subsection{时序分析和时钟频率}
设置时钟周期为10ns,占空比为50\%,在Vivado中综合后进行时序分析。结果显示建立时间约束违例 -5.827ns,保持时间约束富裕0.127ns,关键路径见图 \ref{fig:sta-setup}. 需要注意的是Vivado可能会将一些逻辑比较简单的模块合并到其他模块中,需要在实例化模块时使用(* keep\_hierarchy = ``yes'' *)语句来保留模块层次结构以便更清晰地观察关键路径。

\begin{figure}[H]
    \centering
    \includegraphics[width=\linewidth]{images/STA-Setup.pdf}
    \vspace{-2em}
    \caption{时序分析建立时间的关键路径}
    \label{fig:sta-setup}
\end{figure}

将时钟周期改为16ns,占空比为50\%,重新进行时序分析,结果显示建立时间和保持时间约束均满足。于是使用锁相环IP将原始时钟信号分频至60MHz,作为CPU的时钟。

\subsection{资源占用}
资源占用情况如图 \ref{fig:utilization}. 各资源占用情况均不超过20\%. 其中,LUT主要在DataMem和ALU中被使用,主要负责逻辑运算;FF主要在DataMem和RegisterFile中被使用,主要负责存储;MUX主要在DataMem和RegisterFile中被使用,主要负责选择存储单元;DSP仅在ALU中被使用,负责乘法运算;PLL用于分频器,负责将原始时钟信号分频至60MHz。

\begin{figure}[H]
    \centering
    \subfigure[总体情况]{
        \includegraphics[width=\linewidth]{images/Utilization-Summary.png}
    } \\
    \subfigure[各模块占用情况]{
        \includegraphics[width=\linewidth]{images/Utilization-Hierarchy.png}
    }
    \vspace{-1em}
    \caption{资源占用情况}
    \label{fig:utilization}
\end{figure}
