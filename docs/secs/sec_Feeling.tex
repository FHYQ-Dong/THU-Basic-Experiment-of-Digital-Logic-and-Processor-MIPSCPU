\section{感想体会}

\subsection{开源工作流的便捷性}
在本实验中,除了最后Implementation阶段必须使用Vivado外,在编写Verilog代码和行为级仿真过程中均使用了开源工具如iverilog\footnote{\href{https://github.com/steveicarus/iverilog}{https://github.com/steveicarus/iverilog}}和GTKWave\footnote{\href{https://github.com/gtkwave/gtkwave}{https://github.com/gtkwave/gtkwave}}。这些工具相比Vivado而言较为轻量化、运行速度较快,且可以方便地通过命令行进行操作,极大地提高了实验效率。此外,开源工具的使用也使得代码的可移植性更强,可以在不同的操作系统上运行。

\subsection{仿真测试的重要性}
在本实验中,我通过行为级仿真全面地测试了CPU的功能的正确性,并成功地发现了多个潜在的bug。正是因为在仿真阶段进行了充分的测试,才使得后续阶段能够迅速顺利地完成。
