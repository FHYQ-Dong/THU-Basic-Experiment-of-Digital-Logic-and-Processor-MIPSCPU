\section{源代码}

\subsection{项目文件结构}
\label{subsec:project-structure}
项目文件结构如下。使用iverilog进行仿真时请将工作目录切换到user/的父目录下。
\begin{lstlisting}[language={bash}, numbers=none]
    user/
    ├── src/ # 存放各CPU设计代码
    │   ├── top.v                      # 顶层模块,包括CPU和分频器
    │   ├── ALU.v                      # ALU和ALU控制单元的实现
    │   ├── Control.v                  # 控制单元的实现
    │   ├── CPU.v                      # 将各个模块连接成为一个完整的CPU
    │   ├── Forwarding.v               # 转发单元的实现
    │   ├── Hazard.v                   # 冒险检测单元的实现
    │   ├── Memory.v                   # 指令存储器和数据存储器的实现
    │   ├── PC.v                       # PC寄存器和相关逻辑的实现
    │   ├── Register.v                 # 寄存器堆、级间寄存器的实现
    │   ├── sparse_matmul.asm          # 稀疏矩阵乘法的汇编代码
    │   └── sparse_matmul.inst         # 由Mars将汇编代码翻译而成的机器码
    ├── sim/ # 存放仿真相关代码
    │   ├── asm2inst.py                # 使用python调用Mars将汇编代码转换为机器码
    │   ├── Mars4_5.jar                # Mars仿真器的jar包
    │   ├── printbin.py                # 使用python将二进制文件中的数据打印出来
    │   ├── tb_sparse_matmul_alog.asm  # 仅含有稀疏矩阵乘法算法的汇编代码,用于统计指令数
    │   ├── tb_sparse_matmul_alog.inst # 由Mars将稀疏矩阵乘法算法的汇编代码翻译成的机器码
    │   ├── tb_sparse_matmul_alog.v    # 稀疏矩阵乘法算法测试模块的Verilog代码
    │   ├── tb_sparse_matmul_alog.in   # 二进制文件,稀疏矩阵乘法算法的输入数据
    │   ├── tb_sparse_matmul_alog.out  # 二进制文件,稀疏矩阵乘法算法的输出数据
    │   ├── tb_CPU_inst_1.asm          # CPU测试用例1的汇编代码,包括算术指令和转发
    │   ├── tb_CPU_inst_1.inst         # 由Mars将CPU测试用例1的汇编代码翻译成的机器码
    │   ├── tb_CPU_inst_1.v            # CPU测试用例1的Verilog代码
    │   ├── tb_CPU_inst_2.asm          # CPU测试用例2的汇编代码,包括控制指令和冒险
    │   ├── tb_CPU_inst_2.inst         # 由Mars将CPU测试用例2的汇编代码翻译成的机器码
    │   ├── tb_CPU_inst_2.v            # CPU测试用例2的Verilog代码
    │   ├── tb_CPU_inst_3.asm          # CPU测试用例3的汇编代码,包括稀疏矩阵乘法和BCD7显示
    │   ├── tb_CPU_inst_3.inst         # 由Mars将CPU测试用例3的汇编代码翻译成的机器码
    │   └── tb_CPU_inst_3.v            # CPU测试用例3的Verilog代码
    ├── data/
    │   ├── MIPSCPU.xdc                # Vivado的约束文件
    │   └── bin/
    │       └── top.bin                # Vivado生成的bitstream
    └── ip/ # Vivado生成的ip核
\end{lstlisting}

\subsection{硬件设计代码}
\lstinputlisting[language={Verilog}, caption={user/src/top.v}, label={code:src-top}]{../user/src/top.v}
\lstinputlisting[language={Verilog}, caption={user/src/ALU.v}, label={code:src-ALU}]{../user/src/ALU.v}
\lstinputlisting[language={Verilog}, caption={user/src/Control.v}, label={code:src-Control}]{../user/src/Control.v}
\lstinputlisting[language={Verilog}, caption={user/src/CPU.v}, label={code:src-CPU}]{../user/src/CPU.v}
\lstinputlisting[language={Verilog}, caption={user/src/Forwarding.v}, label={code:src-Forwarding}]{../user/src/Forwarding.v}
\lstinputlisting[language={Verilog}, caption={user/src/Hazard.v}, label={code:src-Hazard}]{../user/src/Hazard.v}
\lstinputlisting[language={Verilog}, caption={user/src/Memory.v}, label={code:src-Memory}]{../user/src/Memory.v}
\lstinputlisting[language={Verilog}, caption={user/src/PC.v}, label={code:src-PC}]{../user/src/PC.v}
\lstinputlisting[language={Verilog}, caption={user/src/Register.v}, label={code:src-Register}]{../user/src/Register.v}
\lstinputlisting[language={[mips]Assembler}, caption={user/src/sparse\_matmul.asm}, label={code:src-sparse-matmul-asm}]{../user/src/sparse_matmul.asm}
\lstinputlisting[caption={user/src/sparse\_matmul.inst}, label={code:src-sparse-matmul-inst}]{../user/src/sparse_matmul.inst}

\subsection{仿真代码}
\lstinputlisting[language={Python}, caption={user/sim/asm2inst.py}, label={code:sim-asm2inst}]{../user/sim/asm2inst.py}
\lstinputlisting[language={Python}, caption={user/sim/printbin.py}, label={code:sim-printbin}]{../user/sim/printbin.py}
\lstinputlisting[language={[mips]Assembler}, caption={user/sim/sparse\_matmul\_mars.asm}, label={code:sim-tb-sparse-matmul-alog-asm}]{../user/sim/tb_sparse_matmul_alog.asm}
\lstinputlisting[language={Verilog}, caption={user/sim/tb\_sparse\_matmul\_alog.v}, label={code:sim-tb-sparse-matmul-alog-v}]{../user/sim/tb_sparse_matmul_alog.v}
\lstinputlisting[language={[mips]Assembler}, caption={user/sim/tb\_CPU\_inst\_1.asm}, label={code:sim-tb-CPU-inst-1-asm}]{../user/sim/tb_CPU_inst_1.asm}
\lstinputlisting[language={Verilog}, caption={user/sim/tb\_CPU\_inst\_1.v}, label={code:sim-tb-CPU-inst-1-v}]{../user/sim/tb_CPU_inst_1.v}
\lstinputlisting[language={[mips]Assembler}, caption={user/sim/tb\_CPU\_inst\_2.asm}, label={code:sim-tb-CPU-inst-2-asm}]{../user/sim/tb_CPU_inst_2.asm}
\lstinputlisting[language={Verilog}, caption={user/sim/tb\_CPU\_inst\_2.v}, label={code:sim-tb-CPU-inst-2-v}]{../user/sim/tb_CPU_inst_2.v}
\lstinputlisting[language={[mips]Assembler}, caption={user/sim/tb\_CPU\_inst\_3.asm}, label={code:sim-tb-CPU-inst-3-asm}]{../user/sim/tb_CPU_inst_3.asm}
\lstinputlisting[language={Verilog}, caption={user/sim/tb\_CPU\_inst\_3.v}, label={code:sim-tb-CPU-inst-3-v}]{../user/sim/tb_CPU_inst_3.v}
\setlength{\columnsep}{50pt}
\twocolumn
\lstinputlisting[caption={user/sim/tb\_sparse\_matmul\_alog.inst}, label={code:sim-tb-sparse-matmul-alog-inst}]{../user/sim/tb_sparse_matmul_alog.inst}
\lstinputlisting[caption={user/sim/tb\_CPU\_inst\_1.inst}, label={code:sim-tb-CPU-inst-1-inst}]{../user/sim/tb_CPU_inst_1.inst}
\lstinputlisting[caption={user/sim/tb\_CPU\_inst\_2.inst}, label={code:sim-tb-CPU-inst-2-inst}]{../user/sim/tb_CPU_inst_2.inst}
\lstinputlisting[caption={user/sim/tb\_CPU\_inst\_3.inst}, label={code:sim-tb-CPU-inst-3-inst}]{../user/sim/tb_CPU_inst_3.inst}
\onecolumn

\subsection{约束文件}
\lstinputlisting[caption={user/data/MIPSCPU.xdc}, label={code:data-MIPSCPU-xdc}]{../user/data/MIPSCPU.xdc}
