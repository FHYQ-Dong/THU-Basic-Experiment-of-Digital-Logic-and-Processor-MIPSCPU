\section{稀疏矩阵乘法的MIPS代码}

稀疏矩阵乘法的MIPS代码主要分为两部分:稀疏矩阵乘法算法和BCD7数码管显示。由于本人选修的是秋季学期的数字逻辑与处理器课程,汇编大作业实现的不是稀疏矩阵乘法算法,故前一部分使用了已有的汇编代码。下面介绍后一部分的实现。

使用BCD7数码管对运算结果矩阵进行逐行逐元素的显示。对于每一个元素,利用视觉暂留效应,通过高频率刷新轮流显示每一位数字,实现人眼观察上各个数码管显示不同数字的效果。锁定BCD7数码管一次刷新的周期为4096个时钟周期,根据实际实现时的时钟频率调整显示每个矩阵元素时的刷新次数。由于结果矩阵的每个元素均在16位二进制数表示的范围内,且为了能够使一个元素一次性在4个BCD7数码管上显示,因此显示时仅显示了每个元素的低16位。

设计上DataMem的BCD7输出的[7:0]控制\{DP,CG,CF,...,CA\}的显示,[11:8]控制数码管\{0,1,2,3\}的使能。数码管的显示结果与BCD7[11:0]的值的对应关系见表 \ref{tab:BCD7-display}.

需要注意的是,显示十六进制数字时,字母部分全部按照大写显示。此时部分字母的显示样式会与部分数字相同,如B和8、D和0等。于是显示这两个字母时会同时将对应数码管的小数点点亮以示区别。这一点已经在表 \ref{tab:BCD7-display} 中体现。

具体代码见Listing \ref{code:src-sparse-matmul-asm}.

\begin{table}[H]
    \centering
    \begin{threeparttable}
    \begin{tabular}{cc|cc}
        \toprule
        数位\tnote{$\dagger$} & BCD7[11:8] & 数字 & BCD7[7:0] \\
        \midrule
        *--- & 0x1 & 0x0 & 0x3F \\
        -*-- & 0x2 & 0x1 & 0x06 \\
        --*- & 0x4 & 0x2 & 0x5B \\
        ---* & 0x8 & 0x3 & 0x4F \\
             &     & 0x4 & 0x66 \\
             &     & 0x5 & 0x6D \\
             &     & 0x6 & 0x7D \\
             &     & 0x7 & 0x07 \\
             &     & 0x8 & 0x7F \\
             &     & 0x9 & 0x6F \\
             &     & 0xA & 0x77 \\
             &     & 0xB & 0xFF \\
             &     & 0xC & 0x39 \\
             &     & 0xD & 0xBF \\
             &     & 0xE & 0x79 \\
             &     & 0xF & 0x71 \\
        \bottomrule
    \end{tabular}
    \begin{tablenotes}
        \footnotesize
        \item[$\dagger$] ``-''表示该数码管不显示数字,``*''表示该数码管显示数字。
    \end{tablenotes}
    \end{threeparttable}
    \caption{BCD7数码管显示结果与BCD7[11:0]的值的对应关系}
    \label{tab:BCD7-display}
\end{table}
