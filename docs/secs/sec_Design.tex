\section{设计方案}

\subsection{支持的指令}
CPU支持如下指令

\begin{table}[H]
    \centering
    \begin{threeparttable}
    \begin{tabular}{ccc}
        \toprule
        R型指令 & I型指令 & J型指令 \\ 
        \midrule
        add  & lw    & j   \\
        addu & sw    & jal \\
        sub  & lui   &     \\
        subu & addi  &     \\
        mul  & addiu &     \\
        and  & andi  &     \\
        or   & ori\tnote{*} & \\
        xor  & slti  &     \\
        nor  & sltiu &     \\
        sll  & beq   &     \\
        srl  & bne   &     \\
        sra  & blez  &     \\
        slt  & bgtz  &     \\
        sltu & bltz  &     \\
        jr   &       &     \\
        jalr &       &     \\
        \bottomrule
    \end{tabular}
    \begin{tablenotes}
        \footnotesize
        \item[*] 非硬性要求实现的指令,但是Mars翻译出的机器码中存在该指令,需要支持。
    \end{tablenotes}
    \end{threeparttable}
    \caption{CPU支持的指令}
    \label{tab:supported-instruction}
\end{table}

\subsection{流水线分级}
CPU设计采用五级流水线,分为取指IF、译码ID、执行EX、存储器访问MEM和写回WB五个阶段,另设有不属于任何一个阶段的转发单元和冒险检测单元。各个阶段的功能如下:
\begin{itemize}
    \item IF:从指令存储器中读取指令,并将程序计数器PC更新为下一条指令的地址。
    \item ID:译码指令,读取寄存器堆中的数据,并进行立即数扩展。
    \item EX:执行算术逻辑运算,计算跳转地址。
    \item MEM:访问数据存储器,读取或写入数据。
    \item WB:将数据写回寄存器堆。
\end{itemize}

\subsection{模块设计}
CPU的设计分为以下几个模块:
\begin{itemize}
    \item 顶层模块top.
    \item CPU模块CPU.
    \item ALU和ALU控制单元ALU, ALUControlUnit.
    \item 寄存器堆RegisterFile.
    \item 指令存储器InstMem.
    \item 数据存储器DataMem.
    \item 控制单元ControlUnit.
    \item 程序计数器PC.
    \item 转发单元ForwardUnit.
    \item 冒险检测单元HazardUnit.
    \item 立即数扩展单元ImmExtendUnit.
    \item 级间寄存器\{IFID,IDEX,EXMEM,\\MEMWB\}Regs.
\end{itemize}
    
\subsection{模块框图}
见图 \ref{fig:cpu-flowchart}.

\begin{figure*}[ht]
    \centering
    \includegraphics[width=0.8\linewidth]{images/Flowchart.pdf}
    \caption{CPU模块框图}
    \label{fig:cpu-flowchart}
\end{figure*}

\subsection{模块设计细节}

\subsubsection{顶层模块top}
顶层模块包括CPU和分频器。分频器使用锁相环IP实现,按照时序分析的结果使CPU在支持的最高时钟/频率下工作。

\subsubsection{CPU模块CPU}
CPU模块将各个子模块连接起来,形成完整的CPU。各个模块之间的简单逻辑也被放在该模块内实现,具体包括:
\begin{itemize}
    \item 将PC+4的高4位与指令的立即数部分和2'h00拼接,形成完整的j指令的跳转地址。
    \item 将PC+4的高4位与指令的立即数部分和2'h00相加,形成完整的branch类型指令的跳转地址。
    \item 选择PC从何处转发jr, jalr指令所需的寄存器数据。
    \item 选择ALUControlUnit从EXMEMRegs转发时转发何值。
    \item 根据RegDst控制信号决定EXMEMRegs.RegWrite\_Addr为何值。
    \item 选择DataMem从何处转发DataMem.WriteData.
\end{itemize}

\subsubsection{ALU和ALU控制单元ALU, ALUControlUnit}
ALU模块根据ALUOp控制信号进行相应的运算。ALU控制单元根据ALUSrc控制信号和各个数据来源决定ALU的输入。

\subsubsection{寄存器堆RegisterFile}
寄存器堆模块实现寄存器的读写操作。其中写入(包括复位)在时钟下降沿完成,以便WB阶段写回的数据可以立即供ID阶段的指令使用。

\subsubsection{指令存储器InstMem}
存储指令的机器码。硬件实现时使用组合逻辑将指令硬编码到指令存储器中;仿真时通过读取参数INST\_FILE指示的文件初始化指令存储器。具体见第 \ref{subsec:behavioral-simulation} 节。

\subsubsection{数据存储器DataMem}
提供对128字节数据存储器的读写操作,以及映射为地址0x40000010的BCD7数码管的写操作。

\subsubsection{控制单元ControlUnit}
指令的操作码生成各个控制信号。具体包括:
\begin{itemize}
    \item PC更新时的选择信号PCSrc.
    \item PC更新时判断是哪一种branch指令的信号Branch\_Type.
    \item 寄存器堆写回使能信号RegWrite.
    \item 寄存器堆写回地址选择信号RegDst.
    \item 寄存器堆写回数据的来源选择信号MemtoReg.
    \item 数据存储器读取使能信号MemRead.
    \item 数据存储器写入使能信号MemWrite.
    \item ALU输入A、输入B的来源选择信号ALUSrcA, ALUSrcB.
    \item ALU操作码ALUOp.
    \item 立即数扩展操作码ExtOp.
\end{itemize}

\subsubsection{程序计数器PC}
存储当前指令的地址,并根据信号进行更新。

\subsubsection{转发单元ForwardUnit}
解决能够通过数据转发解决的数据冒险。转发路径包括:
\begin{itemize}
    \item 从EXMEMRegs和MEMWBRegs的输出转发到ALUControlUnit的输入。
    \item 从MEMWBRegs的输出转发到DataMem的输入,以解决lw后紧接sw对同一地址先读后写的数据冒险。
    \item 从MEMWBRegs的输出转发到EXMEMRegs.RegRtData,以解决lw后隔一条指令然后sw对同一寄存器先读后写的数据冒险。\footnote{这一条转发路径并不是一开始就注意到的。是经过第 \ref{subsubsec:sim-control-instructions} 节中的仿真后添加的。}
    \item 从EXMEMRegs的输出转发到PC输入,以解决在ID阶段处理jr和jalr指令需要使用\$rs寄存器中的值的数据冒险。
\end{itemize}

\subsubsection{冒险检测单元HazardUnit}
通过阻塞流水线来解决无法通过数据转发解决的冒险以及控制冒险,以及确定PC的PCSrc输入信号的值。冒险检测单元的逻辑包括:
\begin{itemize}
    \item load-use冒险:在EX阶段检测,阻塞IF, ID和EX阶段的指令一个周期,等待lw指令的结果写回寄存器堆。
    \item jr或jalr指令需要使用前序第1或第2条指令的结果:在ID阶段检测,阻塞IF和ID阶段的指令一个周期,等待前序指令的结果写回寄存器堆。\footnote{同上。}
    \item 分支指令的控制冒险:在EX阶段检测,清除IF和ID阶段的指令。
    \item 跳转指令的控制冒险:在ID阶段检测,清除IF阶段的指令。
\end{itemize}

\subsubsection{立即数扩展单元ImmExtendUnit}
根据ExtOp控制信号对指令的立即数进行符号扩展、零扩展或lui指令的高位填充。

\subsubsection{级间寄存器\{IFID,IDEX,EXMEM,\\MEMWB\}Regs}
各个阶段之间的级间寄存器。例如IFIDRegs即为IF阶段和ID阶段之间的寄存器,其余类似。用来存储后续阶段需要使用的数据和控制信号。
