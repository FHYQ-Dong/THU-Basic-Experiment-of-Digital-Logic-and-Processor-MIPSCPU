\section{设计方案}

\subsection{支持的指令}
CPU支持如下指令
\begin{table}[H]
    \centering
    \begin{threeparttable}
    \begin{tabular}{ccc}
        \toprule
        R型指令 & I型指令 & J型指令 \\ 
        \midrule
        add  & lw    & j   \\
        addu & sw    & jal \\
        sub  & lui   &     \\
        subu & addi  &     \\
        mul  & addiu &     \\
        and  & andi  &     \\
        or   & ori\tnote{*} & \\
        xor  & slti  &     \\
        nor  & sltiu &     \\
        sll  & beq   &     \\
        srl  & bne   &     \\
        sra  & blez  &     \\
        slt  & bgtz  &     \\
        sltu & bltz  &     \\
        jr   &       &     \\
        jalr &       &     \\
        \bottomrule
    \end{tabular}
    \begin{tablenotes}
        \footnotesize
        \item[*] 非硬性要求实现的指令,但是Mars翻译出的机器码中存在该指令,需要支持。
    \end{tablenotes}
    \end{threeparttable}
    \caption{CPU支持的指令}
    \label{tab:supported-instruction}
\end{table}

\subsection{模块设计}
CPU的设计分为以下几个模块:
\begin{itemize}
    \item \textbf{顶层模块 \lstinline|top|}:包括CPU和分频器,使CPU在合适的时钟频率下工作。
    \item \textbf{ALU和ALU控制单元 \lstinline|ALU, ALUControlUnit|}:实现算术逻辑运算和ALU控制信号的生成。
    \item \textbf{寄存器堆 \lstinline|RegisterFile|}:实现寄存器的读写操作。
    \item \textbf{指令存储器 \lstinline|InstMem|}:存储指令并提供指令读取功能。
    \item \textbf{数据存储器 \lstinline|DataMem|}:存储数据并提供数据读取和写入功能。
    \item \textbf{控制单元 \lstinline|ControlUnit|}:根据指令类型生成控制信号,控制数据流和操作。
    \item \textbf{程序计数器 \lstinline|PC|}:存储当前执行的指令地址,并在每个时钟周期更新。
    \item \textbf{转发单元 \lstinline|ForwardUnit|}:处理数据冒险,通过转发机制解决数据依赖问题。
    \item \textbf{冒险检测单元 \lstinline|HazardUnit|}:检测数据冒险和控制冒险,并生成相应的控制信号。
    \item \textbf{立即数扩展单元 \lstinline|ImmExtendUnit|}:将立即数按需扩展,以便进行后续操作。
    \item \textbf{级间寄存器 \lstinline|\{IFID, IDEX, EXMEM, MEMWB\}Regs|}:存储流水线各级之间的中间结果,确保数据在流水线中正确传递。
\end{itemize}
    
\subsection{模块框图}
见图 \ref{fig:cpu-flowchart}.
\begin{figure*}[h]
    \centering
    \includegraphics[width=0.8\linewidth]{images/Flowchart.pdf}
    \caption{CPU模块框图}
    \label{fig:cpu-flowchart}
\end{figure*}
