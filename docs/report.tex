\documentclass[utf8,twocolumn]{article}
\usepackage{ctex}
\usepackage{amsmath}
\usepackage{circuitikz}
\usepackage{tikz}
\usepackage{graphicx}
\usepackage{hyperref}
\usepackage{geometry}
\usepackage{pdfpages}
\usepackage{booktabs}
\usepackage{subfigure}
\usepackage{float}
\usepackage{amsmath}
\usepackage{amssymb}
\usepackage{mathrsfs}
\usepackage{multirow}
\usepackage{inputenc}
\usepackage{fancyhdr}
\usepackage{listings}
\usepackage[dvipsnames]{xcolor}
\usepackage{fontspec}
\usepackage{multicol}
\usepackage[most]{tcolorbox}
\tcbuselibrary{listings,breakable}

\newfontfamily\codefont[
    Path = fonts/,
    Extension = .ttf,
    UprightFont = *-Regular,
    BoldFont = *-Bold,
    ItalicFont = *-Italic,
    BoldItalicFont = *-BoldItalic,
    Scale = MatchLowercase,
]{SarasaMonoSC}
\hypersetup{
    colorlinks=true,
    linkcolor=black,
    filecolor=black,      
    urlcolor=black,
    citecolor=black,
}
\lstdefinestyle{code}{ 
    basicstyle   = \codefont,
    breaklines   = true,
    breakindent  = 0pt,
    tabsize      = 4,
    keywordstyle = \bfseries\color{NavyBlue},
    emphstyle    = \bfseries\color{Rhodamine},
    commentstyle = \itshape\color{PineGreen},
    stringstyle  = \bfseries\color{Orange!90!black},
    columns      = fixed,
    numbers      = left,
    numbersep    = 2em,
    numberstyle  = \footnotesize,
    frame        = single,
    framesep     = 1em,
    keepspaces   = true,
    showspaces   = false,
    showstringspaces = false,
}
\lstset{style=code}
\lstdefinelanguage[mips]{Assembler}{%
  alsoletter={.\$},
  morestring=[b]",
  morestring=[b]',
  morecomment=[l]\#,
  morekeywords={[1]abs,abs.d,abs.s,add,add.d,add.s,addi,addiu,addu,%
    and,andi,b,bc1f,bc1t,beq,beqz,bge,bgeu,bgez,bgezal,bgt,bgtu,%
    bgtz,ble,bleu,blez,blt,bltu,bltz,bltzal,bne,bnez,break,c.eq.d,%
    c.eq.s,c.le.d,c.le.s,c.lt.d,c.lt.s,ceil.w.d,ceil.w.s,clo,clz,%
    cvt.d.s,cvt.d.w,cvt.s.d,cvt.s.w,cvt.w.d,cvt.w.s,div,div.d,div.s,%
    divu,eret,floor.w.d,floor.w.s,j,jal,jalr,jr,l.d,l.s,la,lb,lbu,%
    ld,ldc1,lh,lhu,li,ll,lui,lw,lwc1,lwl,lwr,madd,maddu,mfc0,mfc1,%
    mfc1.d,mfhi,mflo,mov.d,mov.s,move,movf,movf.d,movf.s,movn,movn.d,%
    movn.s,movt,movt.d,movt.s,movz,movz.d,movz.s,msub,msubu,mtc0,mtc1,%
    mtc1.d,mthi,mtlo,mul,mul.d,mul.s,mulo,mulou,mult,multu,mulu,neg,%
    neg.d,neg.s,negu,nop,nor,not,or,ori,rem,remu,rol,ror,round.w.d,%
    round.w.s,s.d,s.s,sb,sc,sd,sdc1,seq,sge,sgeu,sgt,sgtu,sh,sle,%
    sleu,sll,sllv,slt,slti,sltiu,sltu,sne,sqrt.d,sqrt.s,sra,srav,srl,%
    srlv,sub,sub.d,sub.s,subi,subiu,subu,sw,swc1,swl,swr,syscall,teq,%
    teqi,tge,tgei,tgeiu,tgeu,tlt,tlti,tltiu,tltu,tne,tnei,trunc.w.d,%
    trunc.w.s,ulh,ulhu,ulw,ush,usw,xor,xori},
  morekeywords={[2].align,.ascii,.asciiz,.byte,.data,.double,.extern,%
    .float,.globl,.half,.kdata,.ktext,.set,.space,.text,.word},
  morekeywords={[3]\$0,\$1,\$2,\$3,\$4,\$5,\$6,\$7,\$8,\$9,\$10,\$11,%
    \$12,\$13,\$14,\$15,\$16,\$17,\$18,\$19,\$20,\$21,\$22,\$23,\$24,%
    \$25,\$26,\$27,\$28,\$29,\$30,\$31,%
    \$zero,\$at,\$v0,\$v1,\$a0,\$a1,\$a2,\$a3,\$t0,\$t1,\$t2,\$t3,\$t4,
    \$t5,\$t6,\$t7,\$s0,\$s1,\$s2,\$s3,\$s4,\$s5,\$s6,\$s7,\$t8,\$t9,%
    \$k0,\$k1,\$gp,\$sp,\$fp,\$ra},
}[strings,comments,keywords]
\newtcblisting{CodeBlock}[1][]{
  % ① 先写固定配置 ---------------------------
  listing only,
  breakable,
  enhanced jigsaw,
  colframe = blue!40!black,
  colback  = blue!2,
  arc      = 1mm,
  left=2pt,right=2pt,top=2pt,bottom=2pt,
  boxrule  = 0.4pt,
  listing options = {
    style=code,
    keepspaces=true,
  },
  % ② 再把用户传进来的 *额外键值* 补到最后 ---
  #1
}
\geometry{a4paper, scale=0.8}
\pagestyle{fancy}
\fancyhf{}
\lhead{流水线MIPS处理器的设计报告}
\rhead{无37~董皓彧~2023010659}
\cfoot{---~~\thepage~~---}
\setlength{\columnsep}{20pt}


\title{流水线MIPS处理器的设计报告}
\author{无37~董皓彧~2023010659}
\date{\zhtoday}

\begin{document}

\maketitle
\thispagestyle{fancy}


\section{仿真测试}
\subsection{CPU性能}
\subsubsection{CPI}
将完整的稀疏矩阵乘法汇编程序中的稀疏矩阵乘法部分代码截取出来(Listling \ref{code:sim-matmul}),通过Mars仿真器统计指令数结果为839条(图 \ref{fig:instruction-count})。
\begin{figure}
    \centering
    \includegraphics[width=0.5\linewidth]{images/Instruction-Count.png}
    \caption{Mars仿真器统计指令数}
    \label{fig:instruction-count}
\end{figure}


\onecolumn
\appendix
% \section{源代码}
源代码的组织方式如下:
\begin{lstlisting}[language={bash}, numbers=none]
    user/
    ├── src/ # 存放各个单元的实现代码
    │   ├── top.v              # 顶层模块,包括CPU和分频器
    │   ├── ALU.v              # ALU和ALU控制单元的实现
    │   ├── Control.v          # 控制单元的实现
    │   ├── CPU.v              # 将各个模块连接成为一个完整的CPU
    │   ├── Forwarding.v       # 转发单元的实现
    │   ├── Hazard.v           # 冒险检测单元的实现
    │   ├── Memory.v           # 指令存储器和数据存储器的实现
    │   ├── PC.v               # PC寄存器和相关逻辑的实现
    │   ├── Register.v         # 寄存器堆、级间寄存器的实现
    │   ├── sparse_matmul.asm  # 稀疏矩阵乘法的汇编代码
    │   └── sparse_matmul.inst # 由Mars将汇编代码翻译而成的机器码
    └── sim/ # 存放仿真相关代码
        ├── asm2inst.py                # 使用python调用Mars将汇编代码转换为机器码
        ├── Mars4_5.jar                # Mars仿真器的jar包
        ├── printbin.py                # 使用python将二进制文件中的数据打印出来
        ├── tb_sparse_matmul_mars.asm  # 仅含有稀疏矩阵乘法算法的汇编代码,用于统计指令数
        ├── tb_sparse_matmul_mars.inst # 由Mars将稀疏矩阵乘法算法的汇编代码翻译成的机器码
        ├── tb_sparse_matmul_mars.v    # 稀疏矩阵乘法算法测试模块的Verilog代码
        ├── tb_sparse_matmul_mars.in   # 二进制文件,稀疏矩阵乘法算法的输入数据
        ├── tb_sparse_matmul_mars.out  # 二进制文件,稀疏矩阵乘法算法的输出数据
        ├── tb_CPU_inst_1.asm          # CPU测试用例1的汇编代码,包括算术指令和转发
        ├── tb_CPU_inst_1.inst         # 由Mars将CPU测试用例1的汇编代码翻译成的机器码
        ├── tb_CPU_inst_1.v            # CPU测试用例1的Verilog代码
        ├── tb_CPU_inst_2.asm          # CPU测试用例2的汇编代码,包括控制指令和冒险
        ├── tb_CPU_inst_2.inst         # 由Mars将CPU测试用例2的汇编代码翻译成的机器码
        ├── tb_CPU_inst_2.v            # CPU测试用例2的Verilog代码
        ├── tb_CPU_inst_3.asm          # CPU测试用例3的汇编代码,包括稀疏矩阵乘法和BCD7显示
        ├── tb_CPU_inst_3.inst         # 由Mars将CPU测试用例3的汇编代码翻译成的机器码
        └── tb_CPU_inst_3.v            # CPU测试用例3的Verilog代码
\end{lstlisting}
\lstinputlisting[language={Verilog}, caption={user/src/top.v}, label={code:src-top}]{../user/src/top.v}
\lstinputlisting[language={Verilog}, caption={user/src/ALU.v}, label={code:src-ALU}]{../user/src/ALU.v}
\lstinputlisting[language={Verilog}, caption={user/src/Control.v}, label={code:src-Control}]{../user/src/Control.v}
\lstinputlisting[language={Verilog}, caption={user/src/CPU.v}, label={code:src-CPU}]{../user/src/CPU.v}
\lstinputlisting[language={Verilog}, caption={user/src/Forwarding.v}, label={code:src-Forwarding}]{../user/src/Forwarding.v}
\lstinputlisting[language={Verilog}, caption={user/src/Hazard.v}, label={code:src-Hazard}]{../user/src/Hazard.v}
\lstinputlisting[language={Verilog}, caption={user/src/Memory.v}, label={code:src-Memory}]{../user/src/Memory.v}
\lstinputlisting[language={Verilog}, caption={user/src/PC.v}, label={code:src-PC}]{../user/src/PC.v}
\lstinputlisting[language={Verilog}, caption={user/src/Register.v}, label={code:src-Register}]{../user/src/Register.v}
\lstinputlisting[language={[mips]Assembler}, caption={user/src/sparse\_matmul.asm}, label={code:src-sparse-matmul-asm}]{../user/src/sparse_matmul.asm}
\lstinputlisting[caption={user/src/sparse\_matmul.inst}, label={code:src-sparse-matmul-inst}]{../user/src/sparse_matmul.inst}

\lstinputlisting[language={Python}, caption={user/sim/asm2inst.py}, label={code:sim-asm2inst}]{../user/sim/asm2inst.py}
\lstinputlisting[language={Python}, caption={user/sim/printbin.py}, label={code:sim-printbin}]{../user/sim/printbin.py}
\lstinputlisting[language={[mips]Assembler}, caption={user/sim/sparse\_matmul\_mars.asm}, label={code:sim-tb-sparse-matmul-mars-asm}]{../user/sim/tb_sparse_matmul_mars.asm}
\lstinputlisting[language={Verilog}, caption={user/sim/tb\_sparse\_matmul\_mars.v}, label={code:sim-tb-sparse-matmul-mars-v}]{../user/sim/tb_sparse_matmul_mars.v}
\lstinputlisting[language={[mips]Assembler}, caption={user/sim/tb\_CPU\_inst\_1.asm}, label={code:sim-tb-CPU-inst-1-asm}]{../user/sim/tb_CPU_inst_1.asm}
\lstinputlisting[language={Verilog}, caption={user/sim/tb\_CPU\_inst\_1.v}, label={code:sim-tb-CPU-inst-1-v}]{../user/sim/tb_CPU_inst_1.v}
\lstinputlisting[language={[mips]Assembler}, caption={user/sim/tb\_CPU\_inst\_2.asm}, label={code:sim-tb-CPU-inst-2-asm}]{../user/sim/tb_CPU_inst_2.asm}
\lstinputlisting[language={Verilog}, caption={user/sim/tb\_CPU\_inst\_2.v}, label={code:sim-tb-CPU-inst-2-v}]{../user/sim/tb_CPU_inst_2.v}
\lstinputlisting[language={[mips]Assembler}, caption={user/sim/tb\_CPU\_inst\_3.asm}, label={code:sim-tb-CPU-inst-3-asm}]{../user/sim/tb_CPU_inst_3.asm}
\lstinputlisting[language={Verilog}, caption={user/sim/tb\_CPU\_inst\_3.v}, label={code:sim-tb-CPU-inst-3-v}]{../user/sim/tb_CPU_inst_3.v}
\setlength{\columnsep}{50pt}
\twocolumn
\lstinputlisting[caption={user/sim/tb\_sparse\_matmul\_mars.inst}, label={code:sim-tb-sparse-matmul-mars-inst}]{../user/sim/tb_sparse_matmul_mars.inst}
\lstinputlisting[caption={user/sim/tb\_CPU\_inst\_1.inst}, label={code:sim-tb-CPU-inst-1-inst}]{../user/sim/tb_CPU_inst_1.inst}
\lstinputlisting[caption={user/sim/tb\_CPU\_inst\_2.inst}, label={code:sim-tb-CPU-inst-2-inst}]{../user/sim/tb_CPU_inst_2.inst}
\lstinputlisting[caption={user/sim/tb\_CPU\_inst\_3.inst}, label={code:sim-tb-CPU-inst-3-inst}]{../user/sim/tb_CPU_inst_3.inst}
\onecolumn



\end{document}
